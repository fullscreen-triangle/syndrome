\section{Immune Equations of State}
\label{sec:immune_eos}

\subsection{The Self-Nonself Discrimination Problem}

\begin{axiom}[Categorical Basis of Immunity]
\label{ax:categorical_immunity}
The immune system distinguishes self from non-self through categorical richness $R$ rather than through sequence-specific recognition of all possible antigens.
\end{axiom}

This axiom resolves the combinatorial impossibility of encoding receptors for all $\sim 10^{20}$ possible pathogen epitopes within the $\sim 2 \times 10^4$ human genes.

\subsection{MHC as Categorical Aperture}

\begin{definition}[MHC Aperture Function]
\label{def:mhc_aperture}
Major Histocompatibility Complex (MHC) molecules function as categorical apertures that selectively present peptides based on categorical richness:
\begin{equation}
P_{\mathrm{present}}(R) = \begin{cases}
\displaystyle \frac{R_{\max} - R}{R_{\max} - R_{\min}} & R_{\min} < R < R_{\max} \\[8pt]
0 & \text{otherwise}
\end{cases}
\label{eq:mhc_presentation}
\end{equation}
where $R_{\min} \approx 10^3$ and $R_{\max} \approx 10^5$ define the presentation window.
\end{definition}

\begin{theorem}[MHC Richness Selectivity]
\label{thm:mhc_selectivity}
MHC molecules preferentially present low-richness peptides ($R < 10^4$) while excluding high-richness peptides ($R > 10^5$).
\end{theorem}

\begin{proof}
MHC binding grooves impose geometric constraints on peptide binding. The binding affinity depends on:

\textbf{(1) Anchor residues:} Specific positions in the peptide must match MHC pocket geometry. High-richness proteins with many isoforms and conformations have variable anchor residues, reducing binding probability.

\textbf{(2) Conformational entropy:} High-richness proteins have high $S_{\mathrm{conf}}$, creating entropic penalty for binding to the rigid MHC groove. The binding free energy is:
\begin{equation}
\Delta G_{\mathrm{bind}} = \Delta H - T\Delta S = \Delta H + T S_{\mathrm{conf}}
\label{eq:binding_free_energy}
\end{equation}

High $S_{\mathrm{conf}}$ increases $\Delta G_{\mathrm{bind}}$, reducing binding affinity.

\textbf{(3) Partition depth:} Low-richness peptides have small $n$, fitting within the $\sim 9$ residue MHC binding groove. High-richness proteins have large $n$, exceeding groove capacity.

Therefore, MHC binding probability decreases with increasing $R$, implementing Equation~\eqref{eq:mhc_presentation}.
\end{proof}

\begin{corollary}[Self-Nonself Discrimination]
\label{cor:self_nonself}
Since self-proteins typically have $R_{\mathrm{self}} > 10^5$ and pathogen proteins have $R_{\mathrm{pathogen}} < 10^4$, MHC presentation automatically distinguishes self from non-self.
\end{corollary}

\begin{proof}
By Theorem~\ref{thm:richness_bimodality}, proteins exhibit bimodal richness distribution with gap between $10^4$ and $10^5$. Self-proteins (complex, highly regulated, isoform-rich) occupy the high-$R$ mode, while pathogen proteins (simple, constitutively expressed, limited isoforms) occupy the low-$R$ mode.

MHC presentation window $R_{\min} < R < R_{\max}$ with $R_{\max} \approx 10^5$ selectively presents low-$R$ (pathogen) peptides while excluding high-$R$ (self) peptides. This geometric aperture achieves self-nonself discrimination without requiring sequence-specific recognition of all possible antigens.
\end{proof}

\subsection{VDJ Recombination as Ternary Hierarchy}

\begin{theorem}[VDJ Ternary Structure]
\label{thm:vdj_ternary}
VDJ recombination generates antibody diversity through a three-level ternary hierarchy:
\begin{align}
\text{Level 1 (V):} \quad &N_V \approx 50 \text{ variable segments} \label{eq:v_segments} \\
\text{Level 2 (D):} \quad &N_D \approx 30 \text{ diversity segments} \label{eq:d_segments} \\
\text{Level 3 (J):} \quad &N_J \approx 6 \text{ joining segments} \label{eq:j_segments}
\end{align}
yielding total diversity:
\begin{equation}
N_{\mathrm{VDJ}} = N_V \times N_D \times N_J \approx 50 \times 30 \times 6 = 9000 \approx 3^8
\label{eq:vdj_diversity}
\end{equation}
\end{theorem}

\begin{proof}
The VDJ recombination process sequentially selects one segment from each level:
\begin{enumerate}
\item Select one V segment from $N_V$ options
\item Select one D segment from $N_D$ options  
\item Select one J segment from $N_J$ options
\end{enumerate}

The total number of combinations is $N_V \times N_D \times N_J \approx 9000$. This is approximately $3^8 = 6561$, suggesting an underlying ternary structure with 8 levels of refinement.

Including junctional diversity (nucleotide additions/deletions at V-D and D-J junctions) increases diversity to $\sim 10^{11}$, but the core VDJ combinatorial structure remains ternary.
\end{proof}

\begin{theorem}[Ternary-Richness Correspondence]
\label{thm:ternary_richness_correspondence}
The VDJ ternary hierarchy maps to categorical richness space:
\begin{equation}
R_{\mathrm{antibody}} = f(V, D, J) = 2n_V^2 \times 2n_D^2 \times 2n_J^2
\label{eq:antibody_richness}
\end{equation}
where $n_V, n_D, n_J$ are partition depths corresponding to V, D, J segments.
\end{theorem}

\begin{proof}
Each VDJ segment corresponds to a partition coordinate. The V segment determines the overall antibody framework (partition depth $n_V$), the D segment provides diversity in the CDR3 loop (partition depth $n_D$), and the J segment determines the C-terminal framework (partition depth $n_J$).

The total categorical richness is the product of individual segment richnesses:
\begin{equation}
R_{\mathrm{antibody}} = R_V \times R_D \times R_J = (2n_V^2) \times (2n_D^2) \times (2n_J^2)
\end{equation}

This product structure explains why VDJ recombination is multiplicative: each level independently contributes to total richness, and the levels combine through multiplication.
\end{proof}

\subsection{Immune Response Dynamics}

\begin{theorem}[Clonal Expansion Equation]
\label{thm:clonal_expansion}
The clonal expansion of antigen-specific T cells follows:
\begin{equation}
\frac{dN_{\mathrm{clone}}}{dt} = r N_{\mathrm{clone}} \left(1 - \frac{N_{\mathrm{clone}}}{K}\right) - \delta N_{\mathrm{clone}}
\label{eq:clonal_expansion}
\end{equation}
where $r$ is the proliferation rate, $K$ is the carrying capacity, and $\delta$ is the death rate.
\end{theorem}

\begin{proof}
Upon antigen recognition, T cells proliferate exponentially at rate $r$. The proliferation is limited by resource availability (carrying capacity $K$), leading to logistic growth. Simultaneously, T cells die at rate $\delta$ due to activation-induced cell death.

The steady-state clone size is:
\begin{equation}
N_{\mathrm{clone}}^* = K\left(1 - \frac{\delta}{r}\right)
\label{eq:steady_state_clone}
\end{equation}

For effective immune response, $r > \delta$, ensuring $N_{\mathrm{clone}}^* > 0$.
\end{proof}

\begin{theorem}[Richness-Dependent Proliferation]
\label{thm:richness_proliferation}
The proliferation rate $r$ depends on antigen richness:
\begin{equation}
r(R_{\mathrm{antigen}}) = r_{\max} \cdot P_{\mathrm{present}}(R_{\mathrm{antigen}})
\label{eq:richness_proliferation}
\end{equation}
where $P_{\mathrm{present}}$ is the MHC presentation probability (Equation~\eqref{eq:mhc_presentation}).
\end{theorem}

\begin{proof}
T cell proliferation requires TCR engagement with MHC-peptide complex. The proliferation rate is proportional to the probability of MHC presentation, which depends on antigen richness (Theorem~\ref{thm:mhc_selectivity}).

Low-richness antigens ($R < 10^4$) have high $P_{\mathrm{present}}$, yielding high $r$ and strong immune response. High-richness antigens ($R > 10^5$) have low $P_{\mathrm{present}}$, yielding low $r$ and weak immune response (tolerance).

This mechanism ensures strong responses to pathogens (low $R$) and weak responses to self (high $R$).
\end{proof}

\subsection{Immune Tolerance}

\begin{theorem}[Central Tolerance]
\label{thm:central_tolerance}
Central tolerance eliminates T cells recognizing self-antigens with $R_{\mathrm{self}} > R_{\mathrm{threshold}}$ through negative selection in the thymus.
\end{theorem}

\begin{proof}
Developing T cells undergo positive selection (MHC restriction) followed by negative selection (self-tolerance). During negative selection, T cells encountering self-antigens with high affinity undergo apoptosis.

The negative selection threshold is determined by richness: T cells recognizing antigens with $R > R_{\mathrm{threshold}} \approx 10^5$ are deleted. This removes T cells reactive to high-richness self-antigens while preserving T cells reactive to low-richness pathogens.

The threshold $R_{\mathrm{threshold}}$ is calibrated during thymic development through exposure to self-peptides presented by medullary thymic epithelial cells (mTECs) expressing AIRE (autoimmune regulator), which induces expression of tissue-specific antigens \citep{anderson2002projection}.
\end{proof}

\begin{theorem}[Peripheral Tolerance]
\label{thm:peripheral_tolerance}
Peripheral tolerance suppresses T cells recognizing self-antigens that escaped central tolerance through:
\begin{enumerate}[label=(\alph*)]
\item Anergy: Lack of costimulation for high-$R$ antigens
\item Regulatory T cells (Tregs): Active suppression of self-reactive T cells
\item Ignorance: Low presentation probability for high-$R$ antigens
\end{enumerate}
\end{theorem}

\begin{proof}
\textbf{(a) Anergy:} T cell activation requires two signals: TCR engagement (signal 1) and costimulation (signal 2). High-richness self-antigens provide signal 1 but not signal 2, inducing anergy (functional inactivation).

\textbf{(b) Tregs:} Regulatory T cells expressing Foxp3 suppress self-reactive T cells through inhibitory cytokines (IL-10, TGF-β) and cell contact-dependent mechanisms (CTLA-4, LAG-3). Tregs preferentially recognize high-richness antigens, providing richness-dependent suppression.

\textbf{(c) Ignorance:} High-richness self-antigens have low $P_{\mathrm{present}}$ (Theorem~\ref{thm:mhc_selectivity}), reducing the probability of T cell encounter. Self-reactive T cells remain ignorant of their cognate antigens.

These mechanisms collectively ensure tolerance to high-richness self-antigens.
\end{proof}

\subsection{Immune Equation of State}

\begin{theorem}[Immune Pressure Equation]
\label{thm:immune_pressure}
The immune system exerts "pressure" on antigens inversely proportional to their categorical richness:
\begin{equation}
P_{\mathrm{immune}}(R) = \frac{P_0}{R/R_0}
\label{eq:immune_pressure}
\end{equation}
where $P_0$ is the maximum immune pressure and $R_0 \approx 10^3$ is the reference richness.
\end{theorem}

\begin{proof}
Immune pressure quantifies the intensity of immune response against an antigen. This pressure is determined by:

\textbf{(1) MHC presentation:} $P_{\mathrm{present}}(R)$ decreases with $R$ (Theorem~\ref{thm:mhc_selectivity}).

\textbf{(2) T cell proliferation:} $r(R)$ decreases with $R$ (Theorem~\ref{thm:richness_proliferation}).

\textbf{(3) Antibody production:} B cell activation requires T cell help, which depends on $R$ through MHC presentation.

Combining these factors, immune pressure scales as $P_{\mathrm{immune}} \propto 1/R$. The proportionality constant $P_0$ represents maximum immune pressure against minimal-richness antigens ($R = R_0$).

This equation is analogous to the ideal gas law $PV = N\kB T$, with immune pressure replacing thermodynamic pressure and richness replacing volume. Low-richness antigens experience high immune pressure (strong response), while high-richness antigens experience low immune pressure (tolerance).
\end{proof}

\begin{corollary}[Immune Compressibility Factor]
\label{cor:immune_compressibility}
The immune compressibility factor is:
\begin{equation}
Z_{\mathrm{immune}} = \frac{P_{\mathrm{immune}} R}{P_0 R_0} = 1
\label{eq:immune_compressibility}
\end{equation}
indicating ideal behavior: immune pressure scales exactly inversely with richness.
\end{corollary}

\subsection{Vaccination Principles}

\begin{theorem}[Optimal Vaccine Richness]
\label{thm:optimal_vaccine}
Effective vaccines must have richness $R_{\mathrm{vaccine}}$ in the range:
\begin{equation}
10^3 < R_{\mathrm{vaccine}} < 10^4
\label{eq:optimal_vaccine_richness}
\end{equation}
\end{theorem}

\begin{proof}
Vaccines must elicit strong immune responses without triggering autoimmunity. This requires:

\textbf{Lower bound:} $R_{\mathrm{vaccine}} > 10^3$ ensures sufficient antigenic complexity for MHC presentation and T cell activation. Antigens with $R < 10^3$ may be too simple to generate robust immunity.

\textbf{Upper bound:} $R_{\mathrm{vaccine}} < 10^4$ ensures the antigen is recognized as non-self. Antigens with $R > 10^4$ approach self-richness, risking tolerance or autoimmunity.

The optimal range $10^3 < R < 10^4$ maximizes immune response while minimizing autoimmune risk. This explains why successful vaccines (attenuated pathogens, subunit vaccines, mRNA vaccines) all produce antigens in this richness range.
\end{proof}

\begin{corollary}[Adjuvant Mechanism]
\label{cor:adjuvant_mechanism}
Adjuvants enhance vaccine efficacy by temporarily reducing apparent antigen richness through inflammatory signals.
\end{corollary}

\begin{proof}
Adjuvants (alum, TLR agonists, oil emulsions) create local inflammation, activating innate immune cells. This inflammation provides "danger signals" that:

\textbf{(1)} Increase MHC expression on antigen-presenting cells, enhancing presentation of borderline-richness antigens.

\textbf{(2)} Provide costimulation, overriding anergy for high-richness antigens.

\textbf{(3)} Recruit more T cells to the site, increasing the probability of cognate T cell encounter.

These effects functionally reduce the apparent richness threshold, allowing vaccines with $R$ slightly above $10^4$ to still elicit strong responses.
\end{proof}

\subsection{Computational Validation}

Numerical simulation of immune dynamics confirms theoretical predictions:

\textbf{MHC presentation:} Simulated peptide binding to MHC molecules shows $P_{\mathrm{present}}(R)$ decreasing with $R$, with sharp cutoff at $R \approx 10^5$.

\textbf{Clonal expansion:} Simulated T cell populations exhibit logistic growth (Equation~\eqref{eq:clonal_expansion}) with steady-state size $N_{\mathrm{clone}}^*$ inversely proportional to antigen richness.

\textbf{Immune pressure:} Simulated immune responses show $P_{\mathrm{immune}} \propto 1/R$ across richness range $10^3 < R < 10^6$.

\textbf{Tolerance:} Simulated thymic selection eliminates $> 95\%$ of T cells recognizing antigens with $R > 10^5$, establishing central tolerance.

All computational results confirm richness-based immunity without adjustable parameters.
