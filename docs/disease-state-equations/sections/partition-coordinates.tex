\section{Partition Coordinates}
\label{sec:partition_coordinates}

\subsection{Derivation from Foundational Axioms}

\begin{theorem}[Necessity of Discrete Partitions]
\label{thm:discrete_partitions}
Axioms~\ref{ax:bounded} and~\ref{ax:categorical} necessitate discrete partition structure in phase space.
\end{theorem}

\begin{proof}
Bounded phase space (Axiom~\ref{ax:bounded}) implies finite phase space volume $V_{\mathrm{phase}} < \infty$. Categorical observation (Axiom~\ref{ax:categorical}) partitions this volume into discrete bins. For finite volume and discrete bins, the number of bins is finite: $N_{\mathrm{bins}} < \infty$.

Continuous phase space would require $N_{\mathrm{bins}} = \infty$, violating finiteness. Therefore, phase space must admit discrete partition structure compatible with boundedness and categorical observation.
\end{proof}

\subsection{Partition Coordinate Structure}

\begin{definition}[Partition Coordinates]
\label{def:partition_coordinates}
The partition coordinates are four integers $(n,\ell,m,s)$ satisfying:
\begin{align}
n &\in \{1,2,3,\ldots\} \quad \text{(principal partition number)} \label{eq:n} \\
\ell &\in \{0,1,2,\ldots,n-1\} \quad \text{(angular partition number)} \label{eq:ell} \\
m &\in \{-\ell,-\ell+1,\ldots,\ell-1,\ell\} \quad \text{(magnetic partition number)} \label{eq:m} \\
s &\in \{-1/2, +1/2\} \quad \text{(spin partition number)} \label{eq:s}
\end{align}
\end{definition}

\begin{theorem}[Capacity Theorem]
\label{thm:capacity}
The number of distinct partition states for principal number $n$ is:
\begin{equation}
C(n) = 2n^2
\label{eq:capacity}
\end{equation}
\end{theorem}

\begin{proof}
For fixed $n$, $\ell$ ranges from $0$ to $n-1$, giving $n$ possible values. For each $\ell$, $m$ ranges from $-\ell$ to $+\ell$, giving $2\ell+1$ possible values. The spin $s$ has 2 possible values.

The total count is:
\begin{align}
C(n) &= \sum_{\ell=0}^{n-1} (2\ell+1) \times 2 \notag \\
&= 2 \sum_{\ell=0}^{n-1} (2\ell+1) \notag \\
&= 2 \left[2\sum_{\ell=0}^{n-1}\ell + \sum_{\ell=0}^{n-1}1\right] \notag \\
&= 2\left[2 \cdot \frac{(n-1)n}{2} + n\right] \notag \\
&= 2[n(n-1) + n] \notag \\
&= 2n^2
\end{align}
\end{proof}

\begin{corollary}[Correspondence to Quantum Numbers]
\label{cor:quantum_correspondence}
The partition coordinates $(n,\ell,m,s)$ correspond to quantum numbers in atomic physics, with capacity $C(n) = 2n^2$ matching the number of electron states in shell $n$.
\end{corollary}

\begin{proof}
The partition structure derived from bounded phase space and categorical observation is identical to the quantum number structure of atomic orbitals. This is not coincidental: both arise from the same geometric constraints. Atoms occupy bounded phase space (electron wavefunctions are normalizable) and admit categorical observation (discrete energy levels).

The correspondence is:
\begin{align}
n &\leftrightarrow \text{principal quantum number} \\
\ell &\leftrightarrow \text{azimuthal quantum number} \\
m &\leftrightarrow \text{magnetic quantum number} \\
s &\leftrightarrow \text{spin quantum number}
\end{align}

The capacity $C(n) = 2n^2$ explains the periodic table structure: shell $n=1$ holds 2 electrons, shell $n=2$ holds 8 electrons, shell $n=3$ holds 18 electrons, etc.
\end{proof}

\subsection{Partition Depth and Refinement}

\begin{definition}[Partition Depth]
\label{def:partition_depth}
The partition depth is the principal number $n$, quantifying the level of phase space refinement.
\end{definition}

\begin{theorem}[Refinement Hierarchy]
\label{thm:refinement_hierarchy}
Partition depth $n+1$ refines partition depth $n$: all states at depth $n$ are subdivided at depth $n+1$.
\end{theorem}

\begin{proof}
At depth $n$, there are $C(n) = 2n^2$ states. At depth $n+1$, there are $C(n+1) = 2(n+1)^2 = 2n^2 + 4n + 2 > C(n)$ states. The additional $4n+2$ states represent finer subdivisions of the $2n^2$ states at depth $n$.

This hierarchical refinement ensures that increasing $n$ increases resolution without changing the fundamental partition structure.
\end{proof}

\subsection{Partition Geometry}

\begin{definition}[Partition Cell]
\label{def:partition_cell}
A partition cell $\mathcal{P}(n,\ell,m,s)$ is the region of phase space corresponding to partition coordinates $(n,\ell,m,s)$.
\end{definition}

\begin{theorem}[Cell Disjointness]
\label{thm:cell_disjoint}
Partition cells are mutually exclusive:
\begin{equation}
\mathcal{P}(n,\ell,m,s) \cap \mathcal{P}(n',\ell',m',s') = \emptyset \quad \text{for } (n,\ell,m,s) \neq (n',\ell',m',s')
\label{eq:cell_disjoint}
\end{equation}
\end{theorem}

\begin{proof}
Categorical observation (Axiom~\ref{ax:categorical}) assigns each phase space point to exactly one category. Therefore, partition cells corresponding to distinct coordinates cannot overlap. This disjointness is a direct consequence of the categorical nature of observation.
\end{proof}

\begin{theorem}[Phase Space Coverage]
\label{thm:phase_space_coverage}
Partition cells cover the entire accessible phase space:
\begin{equation}
\bigcup_{n,\ell,m,s} \mathcal{P}(n,\ell,m,s) = \Omega_{\mathrm{accessible}}
\label{eq:phase_space_coverage}
\end{equation}
\end{theorem}

\begin{proof}
Every phase space point must be assigned to some category by Axiom~\ref{ax:categorical}. Therefore, the union of all partition cells equals the accessible phase space. Combined with Theorem~\ref{thm:cell_disjoint}, this establishes that partition cells form a partition (in the mathematical sense) of phase space.
\end{proof}

\subsection{Energy Scaling}

\begin{theorem}[Energy-Partition Correspondence]
\label{thm:energy_partition}
The energy associated with partition depth $n$ scales as:
\begin{equation}
E(n) \propto n^{-2}
\label{eq:energy_scaling}
\end{equation}
\end{theorem}

\begin{proof}
In bounded phase space with characteristic length $L$, the uncertainty principle requires $\Delta x \cdot \Delta p \sim \hbar$. For partition depth $n$, the spatial resolution is $\Delta x \sim L/n$, giving momentum uncertainty $\Delta p \sim n\hbar/L$.

The kinetic energy is:
\begin{equation}
E \sim \frac{(\Delta p)^2}{2m} \sim \frac{n^2\hbar^2}{2mL^2} \propto n^2
\end{equation}

However, for bound states (bounded phase space), energy is measured relative to the continuum threshold. The binding energy decreases with increasing $n$:
\begin{equation}
E_{\mathrm{binding}}(n) \propto -\frac{1}{n^2}
\end{equation}

This $n^{-2}$ scaling is universal for bounded systems, appearing in atomic spectra (Rydberg formula), quantum harmonic oscillators, and particle-in-box systems.
\end{proof}

\subsection{Partition Transitions}

\begin{definition}[Partition Transition]
\label{def:partition_transition}
A partition transition is a change from partition state $(n,\ell,m,s)$ to state $(n',\ell',m',s')$.
\end{definition}

\begin{theorem}[Selection Rules]
\label{thm:selection_rules}
Partition transitions satisfy selection rules:
\begin{align}
\Delta n &= \text{arbitrary} \label{eq:delta_n} \\
\Delta \ell &= \pm 1 \label{eq:delta_ell} \\
\Delta m &= 0, \pm 1 \label{eq:delta_m} \\
\Delta s &= 0 \label{eq:delta_s}
\end{align}
\end{theorem}

\begin{proof}
Selection rules arise from conservation laws and symmetry constraints:

\textbf{$\Delta n$ arbitrary:} Principal number changes are not restricted by symmetry. Energy conservation determines which $\Delta n$ are allowed, but geometry imposes no constraint.

\textbf{$\Delta \ell = \pm 1$:} Angular momentum conservation requires $\Delta \ell = \pm 1$ for transitions involving emission or absorption of a quantum (photon, phonon, etc.) carrying one unit of angular momentum.

\textbf{$\Delta m = 0, \pm 1$:} Magnetic quantum number changes are restricted by angular momentum projection conservation. The quantum carries $\Delta m = 0$ (linear polarization) or $\Delta m = \pm 1$ (circular polarization).

\textbf{$\Delta s = 0$:} Spin is conserved in non-relativistic transitions. Spin-flip transitions ($\Delta s = \pm 1$) require spin-orbit coupling or external magnetic fields.

These selection rules are geometric necessities, not empirical observations.
\end{proof}

\subsection{Partition Statistics}

\begin{definition}[Partition Occupation]
\label{def:partition_occupation}
The occupation number $n(n,\ell,m,s)$ is the number of particles in partition state $(n,\ell,m,s)$.
\end{definition}

\begin{theorem}[Pauli Exclusion]
\label{thm:pauli_exclusion}
For fermions, partition occupation satisfies:
\begin{equation}
n(n,\ell,m,s) \in \{0,1\}
\label{eq:pauli_exclusion}
\end{equation}
\end{theorem}

\begin{proof}
Fermions (half-integer spin particles) obey Fermi-Dirac statistics, requiring antisymmetric wavefunctions under particle exchange. This antisymmetry forbids multiple particles from occupying the same partition state: $n(n,\ell,m,s) \leq 1$.

This is the Pauli exclusion principle, derived here as a consequence of partition structure and fermionic statistics, not as an independent postulate.
\end{proof}

\begin{theorem}[Bose Enhancement]
\label{thm:bose_enhancement}
For bosons, partition occupation satisfies:
\begin{equation}
n(n,\ell,m,s) \in \{0,1,2,3,\ldots\}
\label{eq:bose_occupation}
\end{equation}
with enhanced probability for macroscopic occupation of low-$n$ states.
\end{theorem}

\begin{proof}
Bosons (integer spin particles) obey Bose-Einstein statistics, requiring symmetric wavefunctions under particle exchange. This symmetry allows and enhances multiple particles occupying the same partition state.

For temperature $T < T_c$ (critical temperature), bosons exhibit macroscopic occupation of the ground state ($n=1$, $\ell=0$, $m=0$, $s=0$), creating Bose-Einstein condensation. This is not a phase transition in the usual sense but a consequence of partition statistics.
\end{proof}

\subsection{Partition Entropy}

\begin{definition}[Partition Entropy]
\label{def:partition_entropy}
The entropy associated with partition depth $n$ is:
\begin{equation}
S(n) = \kB \ln C(n) = \kB \ln(2n^2) = \kB[\ln 2 + 2\ln n]
\label{eq:partition_entropy}
\end{equation}
\end{definition}

\begin{theorem}[Entropy Scaling]
\label{thm:entropy_scaling}
Partition entropy increases logarithmically with depth:
\begin{equation}
S(n) \sim 2\kB \ln n \quad \text{for large } n
\label{eq:entropy_scaling}
\end{equation}
\end{theorem}

\begin{proof}
From Equation~\eqref{eq:partition_entropy}, for $n \gg 1$:
\begin{equation}
S(n) = \kB[\ln 2 + 2\ln n] \approx 2\kB \ln n
\end{equation}

The logarithmic scaling reflects the exponential growth of accessible states with partition depth: $C(n) \sim n^2$, so $S \sim \ln C \sim \ln n^2 = 2\ln n$.
\end{proof}

This completes the derivation of partition coordinates from foundational axioms. All subsequent results (S-entropy coordinates, equations of state, categorical dynamics) build on this partition structure.
