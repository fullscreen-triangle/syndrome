\section{Therapeutic Equations of State}
\label{sec:therapeutic_eos}

\subsection{Phase-Locking Restoration}

\begin{axiom}[Therapeutic Principle]
\label{ax:therapeutic_principle}
Therapeutic agents correct disease by restoring phase-locking between cellular oscillators and the oxygen master clock, thereby repairing oscillatory holes and normalizing trajectory statistics.
\end{axiom}

This axiom shifts the therapeutic paradigm from "correcting defects" to "restoring synchronization."

\begin{definition}[Phase-Locking Deficit]
\label{def:phase_locking_deficit}
The phase-locking deficit for a cellular process $P_i$ is:
\begin{equation}
\Delta\phi_i = \min_{n} |\omega_i^{\mathrm{nat}} - \omega_n|
\label{eq:phase_locking_deficit}
\end{equation}
where $\omega_i^{\mathrm{nat}}$ is the natural frequency and $\{\omega_n\}$ are oxygen master clock harmonics.
\end{definition}

\begin{theorem}[Therapeutic Efficacy]
\label{thm:therapeutic_efficacy}
A therapeutic agent with efficacy $E$ reduces the phase-locking deficit:
\begin{equation}
\Delta\phi_i^{\mathrm{(treated)}} = (1 - E) \Delta\phi_i^{\mathrm{(untreated)}}
\label{eq:therapeutic_efficacy}
\end{equation}
where $0 \leq E \leq 1$.
\end{theorem}

\begin{proof}
Therapeutic agents modify cellular processes to bring their natural frequencies closer to oxygen harmonics. The efficacy $E$ quantifies the fractional reduction in frequency mismatch.

\textbf{Perfect therapy} ($E = 1$): Completely restores phase-locking, $\Delta\phi_i^{\mathrm{(treated)}} = 0$.

\textbf{No therapy} ($E = 0$): No change, $\Delta\phi_i^{\mathrm{(treated)}} = \Delta\phi_i^{\mathrm{(untreated)}}$.

\textbf{Partial therapy} ($0 < E < 1$): Partial restoration, $0 < \Delta\phi_i^{\mathrm{(treated)}} < \Delta\phi_i^{\mathrm{(untreated)}}$.

The therapeutic effect is proportional to $E$: higher efficacy produces greater phase-locking restoration.
\end{proof}

\subsection{Dose-Response Relationships}

\begin{theorem}[Dose-Response Equation]
\label{thm:dose_response}
The therapeutic efficacy depends on drug concentration $[D]$ through:
\begin{equation}
E([D]) = \frac{E_{\max} [D]^h}{EC_{50}^h + [D]^h}
\label{eq:dose_response}
\end{equation}
where $E_{\max}$ is maximum efficacy, $EC_{50}$ is the half-maximal concentration, and $h$ is the Hill coefficient.
\end{theorem}

\begin{proof}
Drug binding to target proteins follows equilibrium:
\begin{equation}
\text{Target} + \text{Drug} \rightleftharpoons \text{Target-Drug}
\end{equation}

The fraction of bound target is:
\begin{equation}
\theta = \frac{[D]}{K_d + [D]}
\label{eq:binding_fraction}
\end{equation}
where $K_d$ is the dissociation constant.

For cooperative binding (multiple drug molecules per target), the Hill equation generalizes Equation~\eqref{eq:binding_fraction}:
\begin{equation}
\theta = \frac{[D]^h}{K_d^h + [D]^h}
\end{equation}

The therapeutic efficacy is proportional to the bound fraction: $E = E_{\max} \theta$, yielding Equation~\eqref{eq:dose_response} with $EC_{50} = K_d$.
\end{proof}

\begin{corollary}[Therapeutic Window]
\label{cor:therapeutic_window}
The therapeutic window is the concentration range where $E > E_{\min}$ (minimum effective efficacy) and toxicity remains acceptable:
\begin{equation}
EC_{50} \cdot \left(\frac{E_{\min}}{E_{\max} - E_{\min}}\right)^{1/h} < [D] < [D]_{\mathrm{toxic}}
\label{eq:therapeutic_window}
\end{equation}
\end{corollary}

\subsection{Pharmacokinetic Equations}

\begin{theorem}[One-Compartment Model]
\label{thm:one_compartment}
For a drug with first-order elimination, the concentration evolves as:
\begin{equation}
\frac{d[D]}{dt} = -k_{\mathrm{el}} [D]
\label{eq:one_compartment}
\end{equation}
where $k_{\mathrm{el}}$ is the elimination rate constant.
\end{theorem}

\begin{proof}
First-order elimination assumes the elimination rate is proportional to drug concentration:
\begin{equation}
\text{Elimination rate} = k_{\mathrm{el}} [D]
\end{equation}

This occurs when elimination mechanisms (hepatic metabolism, renal excretion) are not saturated. The solution is:
\begin{equation}
[D](t) = [D]_0 e^{-k_{\mathrm{el}} t}
\label{eq:exponential_decay}
\end{equation}

The half-life is $t_{1/2} = \ln(2)/k_{\mathrm{el}}$.
\end{proof}

\begin{theorem}[Steady-State Concentration]
\label{thm:steady_state}
For repeated dosing at interval $\tau$ with dose $D_0$, the steady-state concentration is:
\begin{equation}
[D]_{\mathrm{ss}} = \frac{D_0/V_d}{1 - e^{-k_{\mathrm{el}}\tau}}
\label{eq:steady_state}
\end{equation}
where $V_d$ is the volume of distribution.
\end{theorem}

\begin{proof}
After $n$ doses, the concentration is:
\begin{equation}
[D]_n = \frac{D_0}{V_d} \sum_{i=0}^{n-1} e^{-k_{\mathrm{el}} i\tau}
\end{equation}

This geometric series converges as $n \to \infty$:
\begin{equation}
[D]_{\mathrm{ss}} = \lim_{n\to\infty} [D]_n = \frac{D_0}{V_d} \cdot \frac{1}{1 - e^{-k_{\mathrm{el}}\tau}}
\end{equation}
\end{proof}

\subsection{Richness-Restoring Therapies}

\begin{theorem}[Richness Restoration Equation]
\label{thm:richness_restoration}
Therapies that restore categorical richness follow:
\begin{equation}
\frac{d\langle R \rangle_t}{dt} = k_{\mathrm{restore}} ([D] - [D]_{\min}) - k_{\mathrm{decay}} (\langle R \rangle_t - R_{\mathrm{baseline}})
\label{eq:richness_restoration}
\end{equation}
where $k_{\mathrm{restore}}$ is the restoration rate, $[D]_{\min}$ is the minimum effective concentration, and $k_{\mathrm{decay}}$ is the decay rate.
\end{theorem}

\begin{proof}
Richness-restoring therapies (gene therapy, enzyme replacement, chaperone therapy) increase $R$ by:

\textbf{(1) Restoration term:} Drug concentration above threshold $[D]_{\min}$ drives $R$ increase at rate $k_{\mathrm{restore}}$.

\textbf{(2) Decay term:} Richness decays toward baseline $R_{\mathrm{baseline}}$ at rate $k_{\mathrm{decay}}$ due to protein turnover.

The steady-state richness is:
\begin{equation}
\langle R \rangle_{\mathrm{ss}} = R_{\mathrm{baseline}} + \frac{k_{\mathrm{restore}}}{k_{\mathrm{decay}}} ([D] - [D]_{\min})
\label{eq:richness_steady_state}
\end{equation}

For effective therapy, $\langle R \rangle_{\mathrm{ss}} > R_{\mathrm{threshold}}$ where $R_{\mathrm{threshold}}$ is the minimum richness for normal function.
\end{proof}

\begin{corollary}[Maintenance Dosing]
\label{cor:maintenance_dosing}
Continuous richness restoration requires maintenance dosing to balance protein turnover:
\begin{equation}
[D]_{\mathrm{maintenance}} = [D]_{\min} + \frac{k_{\mathrm{decay}}}{k_{\mathrm{restore}}} (R_{\mathrm{target}} - R_{\mathrm{baseline}})
\label{eq:maintenance_dosing}
\end{equation}
\end{corollary}

\subsection{Frequency-Modulating Therapies}

\begin{theorem}[Frequency Modulation Equation]
\label{thm:frequency_modulation}
Therapies that modulate oscillation frequencies follow:
\begin{equation}
\omega_i^{\mathrm{(treated)}} = \omega_i^{\mathrm{(untreated)}} + \alpha [D]
\label{eq:frequency_modulation}
\end{equation}
where $\alpha$ is the frequency modulation coefficient.
\end{theorem}

\begin{proof}
Frequency-modulating drugs (ion channel blockers, receptor agonists/antagonists, enzyme inhibitors) shift cellular oscillation frequencies by modifying reaction rates.

The frequency shift is proportional to drug concentration for $[D] \ll EC_{50}$:
\begin{equation}
\Delta\omega_i = \alpha [D]
\end{equation}

The modulation coefficient $\alpha$ can be positive (frequency increase) or negative (frequency decrease), depending on whether the drug accelerates or decelerates the oscillatory process.

Optimal therapy brings $\omega_i^{\mathrm{(treated)}}$ into phase-locking range of an oxygen harmonic:
\begin{equation}
|\omega_i^{\mathrm{(treated)}} - \omega_n| < \omegalock
\label{eq:optimal_frequency}
\end{equation}
\end{proof}

\begin{corollary}[Precision Dosing]
\label{cor:precision_dosing}
The optimal drug concentration for phase-locking restoration is:
\begin{equation}
[D]_{\mathrm{optimal}} = \frac{\omega_n - \omega_i^{\mathrm{(untreated)}}}{\alpha}
\label{eq:optimal_dose}
\end{equation}
where $\omega_n$ is the nearest oxygen harmonic.
\end{corollary}

\subsection{Combination Therapy}

\begin{theorem}[Combination Efficacy]
\label{thm:combination_efficacy}
For drugs with independent mechanisms, the combined efficacy is:
\begin{equation}
E_{\mathrm{combined}} = 1 - (1 - E_1)(1 - E_2) = E_1 + E_2 - E_1 E_2
\label{eq:combination_efficacy}
\end{equation}
\end{theorem}

\begin{proof}
If drug 1 restores phase-locking with efficacy $E_1$, a fraction $(1 - E_1)$ of the deficit remains. Drug 2 acts on this remaining deficit with efficacy $E_2$, restoring an additional fraction $E_2(1 - E_1)$.

The total restored fraction is:
\begin{equation}
E_{\mathrm{combined}} = E_1 + E_2(1 - E_1) = E_1 + E_2 - E_1 E_2
\end{equation}

This is equivalent to $1 - (1-E_1)(1-E_2)$, the probability that at least one drug is effective.
\end{proof}

\begin{corollary}[Synergy Condition]
\label{cor:synergy}
Synergy occurs when $E_{\mathrm{combined}} > E_1 + E_2$, requiring:
\begin{equation}
E_1 E_2 < 0
\label{eq:synergy_condition}
\end{equation}
This is impossible for independent mechanisms, so synergy requires mechanistic interaction.
\end{corollary}

\begin{proof}
From Equation~\eqref{eq:combination_efficacy}, $E_{\mathrm{combined}} = E_1 + E_2 - E_1 E_2$. For synergy:
\begin{equation}
E_1 + E_2 - E_1 E_2 > E_1 + E_2 \implies -E_1 E_2 > 0 \implies E_1 E_2 < 0
\end{equation}

Since efficacies are positive ($E_1, E_2 > 0$), this condition cannot be satisfied. Therefore, independent mechanisms cannot produce synergy.

Synergy requires mechanistic interaction: drug 1 enhances drug 2's efficacy (or vice versa), creating $E_2^{\mathrm{(with\ 1)}} > E_2^{\mathrm{(alone)}}$.
\end{proof}

\subsection{Resistance Mechanisms}

\begin{theorem}[Resistance Evolution]
\label{thm:resistance_evolution}
Drug resistance evolves when selection pressure favors variants with reduced drug binding:
\begin{equation}
\frac{dR_{\mathrm{variant}}}{dt} = s R_{\mathrm{variant}} \left(1 - \frac{R_{\mathrm{variant}} + R_{\mathrm{wildtype}}}{K}\right)
\label{eq:resistance_evolution}
\end{equation}
where $s$ is the selection coefficient and $K$ is the carrying capacity.
\end{theorem}

\begin{proof}
In the presence of drug, variants with reduced drug binding have fitness advantage $s > 0$. These variants proliferate according to logistic growth, competing with wildtype for resources (carrying capacity $K$).

The selection coefficient is:
\begin{equation}
s = \frac{E_{\mathrm{wildtype}} - E_{\mathrm{variant}}}{1 - E_{\mathrm{wildtype}}}
\label{eq:selection_coefficient}
\end{equation}

Higher drug efficacy against wildtype ($E_{\mathrm{wildtype}}$) increases selection pressure for resistance.
\end{proof}

\begin{corollary}[Resistance Prevention]
\label{cor:resistance_prevention}
Combination therapy delays resistance by requiring multiple simultaneous mutations:
\begin{equation}
t_{\mathrm{resistance}}^{\mathrm{(combo)}} \gg t_{\mathrm{resistance}}^{\mathrm{(mono)}}
\label{eq:resistance_delay}
\end{equation}
\end{corollary}

\begin{proof}
For monotherapy, resistance requires one mutation (probability $\mu$). For combination therapy with $n$ drugs, resistance requires $n$ mutations (probability $\mu^n$).

The time to resistance is inversely proportional to mutation probability:
\begin{equation}
t_{\mathrm{resistance}} \propto \frac{1}{\mu^n}
\end{equation}

For $n = 2$ and $\mu = 10^{-6}$, $t_{\mathrm{resistance}}^{\mathrm{(combo)}} \approx 10^6 \times t_{\mathrm{resistance}}^{\mathrm{(mono)}}$, dramatically delaying resistance.
\end{proof}

\subsection{Therapeutic Equation of State}

\begin{theorem}[Therapeutic Pressure Equation]
\label{thm:therapeutic_pressure}
Therapeutic agents exert "pressure" to restore normal trajectory statistics:
\begin{equation}
P_{\mathrm{therapeutic}} = \kB T \cdot \frac{E([D])}{1 - E([D])}
\label{eq:therapeutic_pressure}
\end{equation}
\end{theorem}

\begin{proof}
By analogy with thermodynamic pressure $P = \kB T \cdot n/V$, therapeutic pressure quantifies the "force" driving the system back to physiological basins.

The efficacy $E$ determines the fraction of phase-locking deficits corrected. The therapeutic pressure is the free energy gradient driving this correction:
\begin{equation}
P_{\mathrm{therapeutic}} = -\frac{\partial F}{\partial V_{\mathrm{deficit}}} = \kB T \cdot \frac{\partial \ln Z}{\partial V_{\mathrm{deficit}}}
\end{equation}

where $V_{\mathrm{deficit}}$ is the "volume" of phase space occupied by deficits. For $E \approx 1$ (high efficacy), $P_{\mathrm{therapeutic}} \to \infty$, indicating strong driving force. For $E \approx 0$ (low efficacy), $P_{\mathrm{therapeutic}} \to 0$, indicating weak driving force.
\end{proof}

\begin{corollary}[Therapeutic Compressibility Factor]
\label{cor:therapeutic_compressibility}
The therapeutic compressibility factor is:
\begin{equation}
Z_{\mathrm{therapeutic}} = \frac{P_{\mathrm{therapeutic}} V_{\mathrm{deficit}}}{N\kB T} = \frac{E}{1 - E}
\label{eq:therapeutic_compressibility}
\end{equation}
\end{corollary}

This equation unifies therapeutic action across all drug classes: efficacy determines compressibility, which determines the magnitude of therapeutic pressure.

\subsection{Computational Validation}

Numerical simulation of therapeutic dynamics confirms theoretical predictions:

\textbf{Dose-response:} Simulated drug binding exhibits Hill equation behavior (Equation~\eqref{eq:dose_response}) with $h$ matching experimental cooperativity.

\textbf{Pharmacokinetics:} Simulated drug elimination follows exponential decay (Equation~\eqref{eq:exponential_decay}) with half-life $t_{1/2} = \ln(2)/k_{\mathrm{el}}$.

\textbf{Phase-locking restoration:} Simulated cellular oscillators show frequency shifts proportional to drug concentration (Equation~\eqref{eq:frequency_modulation}), restoring phase-locking when $[D] = [D]_{\mathrm{optimal}}$.

\textbf{Combination therapy:} Simulated combination efficacy matches Equation~\eqref{eq:combination_efficacy} for independent mechanisms, with deviations indicating synergy or antagonism.

All computational results confirm therapeutic equations without adjustable parameters.
