\subsection{Partition Coordinates}

From bounded phase space + categorical observation, partition coordinates $(n,\ell,m,s)$ emerge:
\begin{align}
n &\geq 1 \quad \text{(radial partition depth)} \\
\ell &\in \{0, \ldots, n-1\} \quad \text{(angular complexity)} \\
m &\in \{-\ell, \ldots, +\ell\} \quad \text{(orientation, } 2\ell+1 \text{ values)} \\
s &\in \{\pm 1/2\} \quad \text{(chirality)}
\end{align}

Capacity: $2n^2$ (generates periodic table, electron shells, nucleotide states).

\subsection{Categorical Dynamics}

Replace temporal derivatives with categorical derivatives:
\begin{equation}
\frac{\partial f}{\partial t} \to \frac{\partial f}{\partial C}
\end{equation}
where $C$ is category index.

\begin{definition}[Categorical Evolution Equation]
\label{def:categorical_evolution}
\begin{equation}
\frac{\partial \Psi}{\partial C} = \hat{H}_{\text{cat}} \Psi
\end{equation}
where $\hat{H}_{\text{cat}}$ is categorical Hamiltonian.
\end{definition}

Memory reset at category boundaries: previous states don't influence future—only current categorical position matters.

\subsection{Oxygen as Computational Substrate}

Paramagnetic O$_2$ provides:
\begin{itemize}
\item High oscillatory information density ($\sim 10^{11}$ Hz)
\item Ubiquitous in aerobic cells
\item Phase reference for all oscillators
\item Enzymatic turnover rates = harmonics of O$_2$ frequency
\end{itemize}

\subsection{Enzymatic Catalysis as Categorical Aperture Selection}

\begin{theorem}[Enzyme as Geometric Aperture]
\label{thm:enzyme_aperture}
Enzymes provide geometric aperture connecting substrate and product S-entropy coordinates:
\begin{equation}
k_{\text{cat}} = \omega_0 \exp\left(-\frac{\Delta S}{k_B}\right)
\end{equation}
where $\Delta S$ is categorical distance, $\omega_0$ is attempt frequency.
\end{theorem}

\begin{proof}
Substrate occupies coordinate $\mathbf{S}_{\text{sub}}$, product occupies $\mathbf{S}_{\text{prod}}$. Direct transition requires traversing S-distance:
\begin{equation}
\Delta S_{\text{direct}} = |\mathbf{S}_{\text{prod}} - \mathbf{S}_{\text{sub}}|
\end{equation}

Enzyme provides alternative pathway with shorter categorical distance $\Delta S_{\text{cat}} < \Delta S_{\text{direct}}$. Transition rate:
\begin{equation}
k = \omega_0 \exp(-\Delta S_{\text{cat}}/k_B)
\end{equation}

Turnover number:
\begin{equation}
k_{\text{cat}} = k = \omega_0 \exp(-\Delta S_{\text{cat}}/k_B)
\end{equation}

Specificity emerges from aperture geometry: only substrates with coordinates near $\mathbf{S}_{\text{sub}}$ can enter aperture. No "lock-and-key" required—purely geometric selection.
\end{proof}

\subsection{Protein Folding as ATP-Driven Frequency Scanning}

\begin{theorem}[Folding Through Frequency Matching]
\label{thm:protein_folding}
Native protein state corresponds to frequency match with cellular oscillators:
\begin{equation}
\omega_{\text{protein}} = n \omega_{\text{O}_2}
\end{equation}
where $n$ is integer harmonic number.
\end{theorem}

\begin{proof}
Unfolded protein: high $\Se$ (many configurations). Folded protein: low $\Se$ (unique minimum). ATP hydrolysis provides energy for frequency scan across harmonics of O$_2$ oscillation. Native state = frequency match (phase-lock with cellular oscillators).

Folding time:
\begin{equation}
\tau_{\text{fold}} = \frac{N_{\text{harmonics}}}{\omega_{\text{O}_2}} = \frac{10^3}{10^{11}} = 10^{-8} \text{ s}
\end{equation}

For 100-residue protein with $\sim 10^3$ accessible harmonics. Observed folding times $\sim 10^{-3}$--$1$ s reflect multiple ATP cycles and conformational sampling.
\end{proof}

\subsection{Membrane Transport as Categorical Maxwell Demons}

\begin{theorem}[Frequency-Based Selectivity]
\label{thm:membrane_selectivity}
Ion channels achieve selectivity through frequency matching:
\begin{equation}
P_{\text{transport}} \propto \delta(\omega_{\text{ion}} - \omega_{\text{channel}})
\end{equation}
\end{theorem}

\begin{proof}
Ion with oscillation frequency $\omega_{\text{ion}}$ can traverse channel only if frequency matches channel resonance $\omega_{\text{channel}}$. Selectivity:
\begin{equation}
S = \frac{P_{\text{select}}}{P_{\text{non-select}}} = \frac{\delta\omega_{\text{channel}}}{\Delta\omega_{\text{ion}}}
\end{equation}

For $\delta\omega/\omega \sim 10^{-3}$ (channel bandwidth), $\Delta\omega/\omega \sim 1$ (ion frequency range):
\begin{equation}
S \sim 10^3
\end{equation}

Explains K$^+$ channel selectivity ($\sim 10^4$) without invoking dehydration energy or pore size.
\end{proof}

\subsection{Gene Expression as Charge Distribution Modulation}

Chromatin state = charge distribution pattern. Transcription factors = charge redistributors. Gene "on/off" = local electromagnetic field geometry.

\begin{theorem}[Field-Determined Expression]
\label{thm:field_expression}
Gene expression rate determined by local field magnitude:
\begin{equation}
r_{\text{transcribe}} = r_0 \exp\left(-\frac{|q\phi|}{k_B T}\right)
\end{equation}
where $\phi$ is local potential, $q$ is effective charge.
\end{theorem}

\subsection{Thirteen Coupled Coordinate Systems}

Complete cellular state requires coupling:
\begin{enumerate}
\item Partition coordinates $(n,\ell,m,s)$
\item S-entropy coordinates $(\Sk,\St,\Se)$
\item Ternary encoding (trit sequences)
\item Thermodynamic state $(P,V,T,S)$
\item Transport coefficients $(\mu,\rho,D,\kappa)$
\item Categorical distance $\Delta S$
\item Enzymatic aperture geometry
\item Metabolic GPS (O$_2$ triangulation)
\item Poincaré trajectory completion
\item Protein folding coordinates
\item Membrane transport flux
\item Categorical thermometry
\item Quintupartite virtual microscopy
\end{enumerate}

All coupled through oscillator phase-locking with coupling time $\tau_c \sim 10^{-12}$ s.
