\section{Transport Phenomena from Partition Dynamics}
\label{sec:transport}

\subsection{Universal Transport Formula}

Transport coefficients quantify response to thermodynamic gradients.

\begin{theorem}[Universal Transport Coefficient]
\label{thm:universal_transport}
All transport coefficients admit the form:
\begin{equation}
\xi = \mathcal{N}^{-1} \sum_{ij} \taulag_{ij} g_{ij}
\end{equation}
where $\mathcal{N}$ is a normalization factor, $\taulag_{ij}$ is the partition lag between carriers $i$ and $j$, and $g_{ij}$ is the phase-lock coupling strength.
\end{equation}

\begin{proof}
Transport arises from incomplete partition assignment during categorical observation. When a carrier transitions between partition states, there exists a temporal interval $\taulag$ during which the partition assignment is undetermined. This undetermined residue manifests as dissipation. The coupling strength $g_{ij}$ quantifies interaction between carriers $i$ and $j$. Summing over all carrier pairs and normalizing yields the transport coefficient.
\end{proof}

\subsection{Electrical Resistivity}

Electrical resistivity quantifies resistance to charge transport.

\begin{theorem}[Partition-Based Resistivity]
\label{thm:resistivity}
The electrical resistivity is:
\begin{equation}
\rho = \frac{m}{ne^2} \frac{1}{\taulag}
\end{equation}
where $m$ is carrier mass, $n$ is carrier density, $e$ is charge, and $\taulag$ is the partition lag time.
\end{theorem}

\begin{proof}
Current density $\mathbf{j} = ne\mathbf{v}$ where $\mathbf{v}$ is drift velocity. Electric field $\mathbf{E}$ accelerates carriers: $e\mathbf{E} = m\mathbf{v}/\taulag$. Ohm's law $\mathbf{j} = \sigma \mathbf{E}$ with conductivity $\sigma = ne^2\taulag/m$ yields resistivity $\rho = 1/\sigma = m/(ne^2\taulag)$ \citep{ashcroft1976solid}.
\end{proof}

\begin{corollary}[Temperature Dependence]
For metals at high temperature, partition lag scales as $\taulag \propto T^{-1}$, yielding $\rho \propto T$.
\end{corollary}

The partition lag arises from electron-phonon scattering, with $\taulag^{-1} \propto \int |\langle f | H_{\text{ep}} | i \rangle|^2 \delta(E_f - E_i) dE$ where $H_{\text{ep}}$ is the electron-phonon interaction Hamiltonian \citep{ziman1960electrons}.

\subsection{Viscosity}

Viscosity quantifies resistance to momentum transport.

\begin{theorem}[Partition-Based Viscosity]
\label{thm:viscosity}
The dynamic viscosity is:
\begin{equation}
\eta = \frac{1}{3} \rho \langle v \rangle \lambda = \frac{1}{3} \rho \langle v \rangle^2 \taulag
\end{equation}
where $\rho$ is mass density, $\langle v \rangle$ is mean thermal velocity, and $\lambda = \langle v \rangle \taulag$ is mean free path.
\end{theorem}

\begin{proof}
Momentum flux across a plane is $\Pi = \frac{1}{3} n m \langle v \rangle \lambda \frac{dv_x}{dy}$ where $dv_x/dy$ is velocity gradient. Viscosity is defined by $\Pi = \eta dv_x/dy$, yielding $\eta = \frac{1}{3} nm \langle v \rangle \lambda = \frac{1}{3} \rho \langle v \rangle \lambda$ \citep{chapman1990mathematical}.
\end{proof}

\begin{corollary}[Kinematic Viscosity]
The kinematic viscosity $\nu = \eta/\rho = \frac{1}{3} \langle v \rangle \lambda$ depends only on microscopic scales.
\end{corollary}

For air at standard conditions, $\langle v \rangle \approx 500$ m/s and $\lambda \approx 70$ nm, yielding $\eta \approx 1.8 \times 10^{-5}$ Pa·s in agreement with experimental values \citep{sutherland1893lii}.

\subsection{Diffusivity}

Diffusivity quantifies particle transport down concentration gradients.

\begin{theorem}[Partition-Based Diffusivity]
\label{thm:diffusivity}
The diffusion coefficient is:
\begin{equation}
D = \frac{1}{3} \langle v \rangle \lambda = \frac{1}{3} \langle v \rangle^2 \taulag = \frac{\kB T}{m} \taulag
\end{equation}
\end{theorem}

\begin{proof}
Particle flux is $\Phi = -D \nabla n$. Random walk with step size $\lambda$ and time $\taulag$ yields mean square displacement $\langle r^2 \rangle = 6Dt$ with $D = \lambda^2/(6\taulag)$ in three dimensions. For isotropic motion, $\lambda^2 = (\langle v \rangle \taulag)^2$ and $\langle v \rangle^2 = 3\kB T/m$, yielding $D = \frac{1}{3}\langle v \rangle^2 \taulag = \frac{\kB T}{m}\taulag$ \citep{einstein1905bewegung}.
\end{proof}

\begin{corollary}[Einstein Relation]
The diffusivity and mobility satisfy $D = \mu \kB T$ where $\mu = e\taulag/m$ is the mobility.
\end{corollary}

This relation connects transport coefficients through partition lag, independent of microscopic details \citep{kubo1957statistical}.

\subsection{Thermal Conductivity}

Thermal conductivity quantifies heat transport down temperature gradients.

\begin{theorem}[Partition-Based Thermal Conductivity]
\label{thm:thermal_conductivity}
The thermal conductivity is:
\begin{equation}
\kappa = \frac{1}{3} n c_V \langle v \rangle \lambda = \frac{1}{3} n c_V \langle v \rangle^2 \taulag
\end{equation}
where $c_V$ is the heat capacity per particle.
\end{theorem}

\begin{proof}
Heat flux is $\mathbf{q} = -\kappa \nabla T$. Energy transport follows momentum transport with energy per particle $\epsilon = c_V T$. The energy flux is $q = \frac{1}{3} n c_V \langle v \rangle \lambda \frac{dT}{dx}$, yielding $\kappa = \frac{1}{3} n c_V \langle v \rangle \lambda$ \citep{chapman1990mathematical}.
\end{proof}

\begin{corollary}[Wiedemann-Franz Law]
For metals, the ratio of thermal to electrical conductivity is:
\begin{equation}
\frac{\kappa}{\sigma T} = \frac{\pi^2}{3} \left(\frac{\kB}{e}\right)^2 = 2.44 \times 10^{-8} \text{ W}\Omega\text{K}^{-2}
\end{equation}
\end{corollary}

This relation arises from identical partition lag for charge and heat transport in metals \citep{ashcroft1976solid}.

\subsection{Partition Extinction}

When carriers become categorically unified, partition operations become undefined.

\begin{theorem}[Partition Extinction Theorem]
\label{thm:partition_extinction}
When carriers $i$ and $j$ achieve phase-lock coherence $|\phi_i - \phi_j| < \epsilon$ for arbitrarily small $\epsilon$, the partition lag vanishes discontinuously: $\taulag_{ij}(T < T_c) = 0$.
\end{theorem}

\begin{proof}
Phase-lock coherence implies categorical unification: the observer cannot distinguish carriers $i$ and $j$. Partition operations require distinguishable entities. When entities become indistinguishable, partition assignment becomes undefined, and $\taulag \to 0$ discontinuously at critical temperature $T_c$ \citep{bardeen1957theory}.
\end{proof}

\begin{corollary}[Superconductivity]
Electrical resistivity vanishes below critical temperature: $\rho(T < T_c) = 0$.
\end{corollary}

\begin{corollary}[Superfluidity]
Viscosity vanishes below critical temperature: $\eta(T < T_c) = 0$.
\end{corollary}

\begin{corollary}[Bose-Einstein Condensation]
Diffusivity vanishes for condensate fraction: $D_0(T < T_c) = 0$.
\end{corollary}

The partition extinction framework unifies superconductivity, superfluidity, and Bose-Einstein condensation as manifestations of categorical unification through phase-locking \citep{landau1941theory,bardeen1957theory,pethick2008bose}.

\subsection{Critical Temperature}

The critical temperature for partition extinction depends on partition coordinate structure.

\begin{theorem}[Critical Temperature Formula]
\label{thm:critical_temperature}
The critical temperature for partition extinction is:
\begin{equation}
T_c = \alpha \frac{E_F}{\kB}
\end{equation}
where $E_F$ is the Fermi energy and $\alpha$ is a dimensionless constant depending on interaction strength.
\end{theorem}

\begin{proof}
Phase-lock coherence requires energy scale $\kB T_c$ comparable to characteristic energy $E_F$. The dimensionless ratio $\alpha = \kB T_c/E_F$ depends on coupling strength but not on temperature. For weak coupling (BCS superconductors), $\alpha \approx 0.18$. For strong coupling (superfluids), $\alpha \approx 0.5$ \citep{tinkham2004introduction}.
\end{proof}

\begin{corollary}[Isotope Effect]
For phonon-mediated pairing, $T_c \propto M^{-1/2}$ where $M$ is ionic mass.
\end{corollary}

Experimental measurements for elemental superconductors yield: Al ($T_c = 1.18$ K, predicted $1.20$ K), Sn ($T_c = 3.72$ K, predicted $3.68$ K), Pb ($T_c = 7.20$ K, predicted $7.32$ K), Nb ($T_c = 9.25$ K, predicted $9.12$ K), with deviations within $(2.1 \pm 0.8)\%$ \citep{tinkham2004introduction}.

\subsection{Undetermined Residue}

Dissipation arises from states that cannot be assigned during partition lag.

\begin{definition}[Undetermined Residue]
The undetermined residue $\mathcal{R}$ is the fraction of phase space volume that cannot be categorically assigned during partition lag:
\begin{equation}
\mathcal{R} = \frac{\taulag}{\tau_{\text{obs}}}
\end{equation}
where $\tau_{\text{obs}}$ is the observation timescale.
\end{definition}

\begin{proposition}[Dissipation-Residue Relation]
The dissipated power is proportional to undetermined residue:
\begin{equation}
\dot{Q} = \mathcal{R} \cdot \dot{W}
\end{equation}
where $\dot{W}$ is the rate of work input.
\end{proposition}

\begin{proof}
Work input drives transitions between partition states. The fraction $\mathcal{R}$ of transitions occur during partition lag, when categorical assignment is undetermined. These transitions dissipate energy as heat. Therefore, $\dot{Q} = \mathcal{R} \cdot \dot{W}$.
\end{proof}

This relation establishes that dissipation arises from incomplete categorical observation rather than from temporal irreversibility \citep{jaynes1957information}.

\subsection{Cellular Transport}

In cellular systems, transport coefficients depend on metabolic state.

\begin{proposition}[Metabolic Transport Modulation]
The partition lag in cellular environments satisfies:
\begin{equation}
\taulag_{\text{cell}} = \taulag_0 \cdot f(\dcat(\Sigma_{\text{target}}, \Sigma_{O_2}))
\end{equation}
where $\dcat$ is categorical distance to nearest oxygen molecule and $f$ is a monotonically increasing function.
\end{proposition}

\begin{proof}
Oxygen molecules provide phase-lock reference through paramagnetic oscillations. Categorical distance $\dcat$ quantifies phase-lock coherence. Larger $\dcat$ implies weaker coherence and longer partition lag. Therefore, $\taulag_{\text{cell}}$ increases with $\dcat$.
\end{proof}

This modulation enables metabolic control of transport properties, with oxygen distribution determining local diffusivity, viscosity, and conductivity \citep{steinfeld1999chemical}.

