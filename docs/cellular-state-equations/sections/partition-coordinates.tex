\section{Partition Coordinate Structure}
\label{sec:partition_coordinates}

\subsection{Geometric Derivation}

Consider a bounded spherical phase space with radius $R < \infty$. Categorical observation partitions this space into nested shells indexed by depth $n \geq 1$. Within each shell, angular structure admits further partitioning.

\begin{definition}[Partition Coordinates]
A state in bounded spherical phase space is characterized by four coordinates:
\begin{itemize}[nosep]
\item Depth $n \in \NN$, $n \geq 1$: radial partition index
\item Complexity $\ell \in \{0,1,\ldots,n-1\}$: angular momentum quantum number
\item Orientation $m \in \{-\ell,-\ell+1,\ldots,+\ell\}$: magnetic quantum number
\item Chirality $s \in \{-\tfrac{1}{2},+\tfrac{1}{2}\}$: spin quantum number
\end{itemize}
\end{definition}

The constraint $\ell < n$ arises from geometric necessity: angular complexity cannot exceed radial depth in spherically symmetric partitioning.

\begin{theorem}[Capacity Theorem]
\label{thm:capacity}
The number of distinguishable states at partition depth $n$ is exactly $C(n) = 2n^2$.
\end{theorem}

\begin{proof}
For fixed $n$, the complexity $\ell$ ranges from $0$ to $n-1$. For each $\ell$, orientation $m$ admits $2\ell+1$ values. Chirality $s$ admits $2$ values. The total count is:
\begin{equation}
C(n) = \sum_{\ell=0}^{n-1} (2\ell+1) \times 2 = 2 \sum_{\ell=0}^{n-1} (2\ell+1) = 2 \left[ 2 \frac{(n-1)n}{2} + n \right] = 2n^2
\end{equation}
\end{proof}

\begin{corollary}[Cumulative Capacity]
The total number of states up to depth $n$ is $\sum_{k=1}^{n} C(k) = \frac{2n(n+1)(2n+1)}{6}$.
\end{corollary}

\subsection{Selection Rules}

Transitions between partition states obey geometric constraints.

\begin{theorem}[Partition Selection Rules]
\label{thm:selection_rules}
A transition from $(n,\ell,m,s)$ to $(n',\ell',m',s')$ is geometrically allowed if and only if:
\begin{align}
\Delta \ell &= \ell' - \ell \in \{-1,0,+1\} \label{eq:delta_ell} \\
\Delta m &= m' - m \in \{-1,0,+1\} \label{eq:delta_m} \\
\Delta s &= s' - s \in \{-1,0,+1\} \label{eq:delta_s}
\end{align}
\end{theorem}

\begin{proof}
Categorical observation with finite resolution distinguishes states differing by at most one partition unit. Transitions spanning multiple partition units require intermediate states, implying that single-step transitions satisfy $|\Delta \ell| \leq 1$, $|\Delta m| \leq 1$, and $|\Delta s| \leq 1$. The depth $n$ may change arbitrarily as radial transitions involve different constraint.
\end{proof}

\subsection{Pauli Exclusion}

The partition coordinate structure imposes occupancy constraints.

\begin{theorem}[Pauli Exclusion Principle]
\label{thm:pauli}
No two indistinguishable entities can occupy the same partition state $(n,\ell,m,s)$ simultaneously.
\end{theorem}

\begin{proof}
Categorical observation assigns entities to partition states. If two indistinguishable entities occupy the same state, the observer cannot distinguish them, violating the premise that they are separate entities. Therefore, indistinguishable entities must occupy distinct partition states.
\end{proof}

\subsection{Partition Signatures}

Multi-entity systems admit compact representation through partition signatures.

\begin{definition}[Partition Signature]
For a system of $N$ entities occupying partition states $\{(n_i,\ell_i,m_i,s_i)\}_{i=1}^{N}$, the partition signature is the multiset $\Sigma = \{\!(n_1,\ell_1,m_1,s_1), \ldots, (n_N,\ell_N,m_N,s_N)\!\}$.
\end{definition}

\begin{proposition}[Signature Uniqueness]
Two systems with identical partition signatures are categorically indistinguishable.
\end{proposition}

\begin{proof}
The partition signature encodes all distinguishable information accessible through categorical observation. Systems with identical signatures produce identical measurement outcomes, rendering them categorically indistinguishable.
\end{proof}

\subsection{Energy Scaling}

Partition coordinates relate to energy through geometric scaling.

\begin{proposition}[Energy-Coordinate Relation]
\label{prop:energy_scaling}
The energy associated with partition state $(n,\ell,m,s)$ scales as $E_n \propto n^{-2}$ for bound systems.
\end{proposition}

\begin{proof}
Bounded phase space with finite extent $R$ imposes wavelength quantization $\lambda_n \propto R/n$. Energy scales as $E \propto \lambda^{-2}$, yielding $E_n \propto n^{-2}$.
\end{proof}

This $n^{-2}$ scaling reproduces the Rydberg formula for hydrogen-like atoms without invoking Schrödinger's equation \citep{rydberg1890recherches,bohr1913constitution}.

\subsection{Hyperfine Structure}

Interaction between electronic and nuclear partition coordinates produces fine structure.

\begin{definition}[Hyperfine Splitting]
The energy shift due to coupling between electronic angular momentum $\mathbf{J}$ and nuclear angular momentum $\mathbf{I}$ is:
\begin{equation}
\Delta E_{\text{hf}} = \frac{A}{2} \left[ F(F+1) - J(J+1) - I(I+1) \right]
\end{equation}
where $F = J + I$ is the total angular momentum and $A$ is the hyperfine coupling constant.
\end{definition}

The hyperfine splitting arises from partition coordinate coupling rather than from temporal dynamics, with $A$ determined by overlap of electronic and nuclear partition distributions \citep{woodgate1980elementary}.

\subsection{Periodic Structure}

The capacity sequence $C(n) = 2n^2$ generates periodic structure in multi-entity systems.

\begin{theorem}[Periodic Table Structure]
\label{thm:periodic}
For a system of $N$ identical fermions filling partition states sequentially, shell closures occur at $N = 2, 10, 28, 60, 110, 182, \ldots$ corresponding to cumulative capacities $\sum_{k=1}^{n} 2k^2$.
\end{theorem}

\begin{proof}
Pauli exclusion (Theorem~\ref{thm:pauli}) requires distinct states for each fermion. Filling states in order of increasing $n$, then $\ell$, then $m$, then $s$ produces shell closures when $N$ equals cumulative capacity. For $n=1$: $C(1)=2$. For $n=2$: $C(1)+C(2)=2+8=10$. For $n=3$: $C(1)+C(2)+C(3)=2+8+18=28$. The pattern continues as $\sum_{k=1}^{n} 2k^2$.
\end{proof}

The sequence $2, 10, 28, 60, 110, 182$ corresponds to noble gas electron configurations (He, Ne, Ar+8, Kr+32, Xe+54, Rn+86), though exact correspondence requires accounting for $\ell$-dependent energy shifts \citep{scerri2007periodic}.

\subsection{Coordinate Transformations}

Partition coordinates admit transformations to alternative representations.

\begin{definition}[Cardinal Coordinates]
The cardinal transformation maps partition coordinates to three-dimensional vectors:
\begin{equation}
\mathbf{c}(n,\ell,m,s) = \left( \frac{n}{\sum_i n_i}, \frac{\ell}{\sum_i \ell_i}, \frac{m}{\sum_i m_i} \right)
\end{equation}
normalizing by total system content.
\end{definition}

\begin{proposition}[Trajectory Mapping]
A sequence of partition signatures $\{\Sigma_1, \Sigma_2, \ldots, \Sigma_K\}$ maps to a trajectory $\{\mathbf{c}_1, \mathbf{c}_2, \ldots, \mathbf{c}_K\}$ in cardinal coordinate space.
\end{proposition}

This mapping enables geometric analysis of partition coordinate evolution, with trajectories in cardinal space representing temporal sequences of partition configurations.

\subsection{Measurement Correspondence}

Partition coordinates correspond to measurable physical quantities.

\begin{proposition}[Mass-Coordinate Relation]
\label{prop:mass_coordinate}
For molecular systems, the mass-to-charge ratio satisfies:
\begin{equation}
\frac{m}{z} = \sum_{i} \frac{A_i}{z_i} \left( 1 + \delta_{n_i,\ell_i,m_i,s_i} \right)
\end{equation}
where $A_i$ is atomic mass, $z_i$ is charge state, and $\delta_{n_i,\ell_i,m_i,s_i}$ is the partition correction depending on occupied states.
\end{equation}

The partition correction $\delta$ accounts for binding energy shifts arising from partition coordinate occupancy, with typical magnitude $|\delta| \sim 10^{-6}$ for organic molecules \citep{gross2017mass}.

\begin{proposition}[Spectral Line Positions]
Spectral line frequencies correspond to partition coordinate transitions:
\begin{equation}
\nu_{if} = \frac{E_i - E_f}{h} = \frac{R_\infty c}{h} \left( \frac{1}{n_f^2} - \frac{1}{n_i^2} \right) + \Delta \nu_{\ell,m,s}
\end{equation}
where $R_\infty$ is the Rydberg constant and $\Delta \nu_{\ell,m,s}$ accounts for fine structure.
\end{proposition}

Experimental measurements of spectral lines extract partition coordinates through inversion of this relation, with precision limited by instrumental resolution \citep{herzberg1950molecular}.

