\section{Poincaré Computing Framework}
\label{sec:poincare}

\subsection{Computational Paradigm}

Computation in bounded phase space corresponds to trajectory completion.

\begin{definition}[Poincaré Computation]
A computation is a trajectory $\gamma: [0,T] \to \Sspace$ in S-entropy space satisfying:
\begin{enumerate}[nosep]
\item Initial condition: $\gamma(0) = \Scoord_0$
\item Constraint satisfaction: $\mathcal{C}(\gamma) = \text{true}$
\item Recurrence: $\|\gamma(T) - \Scoord_0\| < \epsilon$
\end{enumerate}
where $\mathcal{C}$ is a constraint predicate and $\epsilon$ is convergence tolerance.
\end{definition}

\begin{theorem}[Computation-Recurrence Equivalence]
\label{thm:computation_recurrence}
A trajectory $\gamma$ solves the computational problem specified by $\mathcal{C}$ if and only if $\gamma$ satisfies Poincaré recurrence and $\mathcal{C}(\gamma) = \text{true}$.
\end{theorem}

\begin{proof}
Necessity: If $\gamma$ solves the problem, it must satisfy constraints $\mathcal{C}$ and return to initial state (recurrence). Sufficiency: If $\gamma$ satisfies $\mathcal{C}$ and recurrence, it represents a valid solution trajectory. The Poincaré recurrence theorem guarantees existence of such trajectories for measure-preserving dynamics on bounded phase space \citep{poincare1890probleme,katok1995introduction}.
\end{proof}

\subsection{Identity Unification}

In Poincaré computing, memory address, processor state, and semantic content are unified.

\begin{theorem}[Identity Unification]
\label{thm:identity_unification}
For a computational state $\Scoord \in \Sspace$, the memory address $\mathcal{A}(\Scoord)$, processor state $\mathcal{P}(\Scoord)$, and semantic content $\mathcal{M}(\Scoord)$ are simultaneous projections of the same categorical state.
\end{theorem}

\begin{proof}
The S-entropy coordinate $\Scoord = (\Sk, \St, \Se)$ uniquely specifies position in phase space. Memory address is determined by position: $\mathcal{A}(\Scoord) = \Scoord$. Processor state is determined by current position and local dynamics: $\mathcal{P}(\Scoord) = \nabla \Scoord$. Semantic content is determined by partition coordinates at $\Scoord$: $\mathcal{M}(\Scoord) = \{(n,\ell,m,s)\}|_{\Scoord}$. All three are functions of the same state $\Scoord$, establishing identity unification.
\end{proof}

\begin{corollary}[Von Neumann Architecture Elimination]
Poincaré computing eliminates the distinction between code and data: both are encoded in the same S-entropy coordinate space.
\end{corollary}

This unification resolves the von Neumann bottleneck by eliminating memory-processor separation \citep{backus1978can}.

\subsection{Complexity Measure}

Computational complexity is quantified by trajectory structure in S-entropy space.

\begin{definition}[Categorical Complexity]
The categorical complexity of a computation is the number of distinct categorical states visited by the trajectory:
\begin{equation}
\mathcal{C}_{\text{cat}}(\gamma) = |\{\mathcal{C}_i : \exists t \in [0,T], \gamma(t) \in \mathcal{C}_i\}|
\end{equation}
\end{definition}

\begin{theorem}[Complexity-Recurrence Time]
\label{thm:complexity_time}
The recurrence time $T$ scales with categorical complexity as:
\begin{equation}
T \sim \mathcal{C}_{\text{cat}}(\gamma) \cdot \tau_{\text{step}}
\end{equation}
where $\tau_{\text{step}}$ is the characteristic time for categorical transitions.
\end{theorem}

\begin{proof}
Each categorical state requires time $\tau_{\text{step}}$ for transition. Visiting $\mathcal{C}_{\text{cat}}$ distinct states requires total time $T \sim \mathcal{C}_{\text{cat}} \cdot \tau_{\text{step}}$. The scaling is linear for deterministic trajectories without backtracking.
\end{proof}

\begin{corollary}[Polynomial Time]
A problem is solvable in polynomial time if $\mathcal{C}_{\text{cat}}(\gamma) = \mathcal{O}(n^k)$ for input size $n$ and constant $k$.
\end{corollary}

\subsection{Non-Halting Dynamics}

Poincaré computing admits non-halting trajectories with emergent memory.

\begin{definition}[Persistent Trajectory]
A trajectory $\gamma: [0,\infty) \to \Sspace$ is persistent if it never satisfies recurrence: $\forall T > 0, \|\gamma(T) - \Scoord_0\| \geq \epsilon$.
\end{definition}

\begin{theorem}[Persistent Trajectory Measure]
\label{thm:persistent_measure}
The set of initial conditions $\Scoord_0$ generating persistent trajectories has measure zero in $\Sspace$.
\end{theorem}

\begin{proof}
The Poincaré recurrence theorem states that for measure-preserving dynamics on bounded phase space, almost every trajectory returns arbitrarily close to its initial state \citep{poincare1890probleme}. Therefore, persistent trajectories (those that never return) form a set of measure zero.
\end{proof}

\begin{corollary}[Generic Halting]
For generic initial conditions, all computations eventually halt (satisfy recurrence).
\end{corollary}

\subsection{Emergent Memory}

Memory emerges from trajectory history in S-entropy space.

\begin{definition}[Trajectory Memory]
The memory $\mathcal{M}(t)$ at time $t$ is the set of categorical states visited up to time $t$:
\begin{equation}
\mathcal{M}(t) = \{\mathcal{C}_i : \exists t' \in [0,t], \gamma(t') \in \mathcal{C}_i\}
\end{equation}
\end{definition}

\begin{proposition}[Memory Growth]
The memory size grows as:
\begin{equation}
|\mathcal{M}(t)| \leq \min\left(t/\tau_{\text{step}}, |\Sspace|/\epsilon^3\right)
\end{equation}
where $|\Sspace|/\epsilon^3$ is the number of distinguishable cells in $\Sspace$ at resolution $\epsilon$.
\end{proposition}

\begin{proof}
Each time step $\tau_{\text{step}}$ can add at most one new categorical state to memory, yielding $|\mathcal{M}(t)| \leq t/\tau_{\text{step}}$. The total number of distinguishable states is bounded by phase space volume divided by cell volume: $|\Sspace|/\epsilon^3 = 1/\epsilon^3$ for $\Sspace = [0,1]^3$. Therefore, $|\mathcal{M}(t)| \leq \min(t/\tau_{\text{step}}, 1/\epsilon^3)$.
\end{proof}

\begin{corollary}[Memory Saturation]
Memory saturates at $|\mathcal{M}_{\text{max}}| = 1/\epsilon^3$ after time $T_{\text{sat}} = \tau_{\text{step}}/\epsilon^3$.
\end{corollary}

\subsection{Constraint Satisfaction}

Computational constraints are encoded as geometric constraints in S-entropy space.

\begin{definition}[Constraint Manifold]
A constraint $\mathcal{C}$ defines a manifold $\mathcal{M}_{\mathcal{C}} \subset \Sspace$ of allowed states:
\begin{equation}
\mathcal{M}_{\mathcal{C}} = \{\Scoord \in \Sspace : \mathcal{C}(\Scoord) = \text{true}\}
\end{equation}
\end{definition}

\begin{theorem}[Constrained Trajectory]
\label{thm:constrained_trajectory}
A trajectory $\gamma$ satisfies constraint $\mathcal{C}$ if and only if $\gamma(t) \in \mathcal{M}_{\mathcal{C}}$ for all $t \in [0,T]$.
\end{theorem}

\begin{proof}
By definition, $\mathcal{C}(\Scoord) = \text{true}$ if and only if $\Scoord \in \mathcal{M}_{\mathcal{C}}$. A trajectory satisfies $\mathcal{C}$ if $\mathcal{C}(\gamma(t)) = \text{true}$ for all $t$, which is equivalent to $\gamma(t) \in \mathcal{M}_{\mathcal{C}}$ for all $t$.
\end{proof}

\begin{corollary}[Multiple Constraints]
For constraints $\{\mathcal{C}_1, \ldots, \mathcal{C}_k\}$, the trajectory must remain in the intersection $\bigcap_{i=1}^{k} \mathcal{M}_{\mathcal{C}_i}$.
\end{corollary}

\subsection{Solution Uniqueness}

Under generic conditions, computational problems admit unique solutions.

\begin{theorem}[Generic Uniqueness]
\label{thm:generic_uniqueness}
For generic constraint manifolds $\mathcal{M}_{\mathcal{C}}$ and initial conditions $\Scoord_0$, the trajectory $\gamma$ satisfying $\gamma(0) = \Scoord_0$, $\gamma(t) \in \mathcal{M}_{\mathcal{C}}$, and $\|\gamma(T) - \Scoord_0\| < \epsilon$ is unique.
\end{theorem}

\begin{proof}
The constraint manifold $\mathcal{M}_{\mathcal{C}}$ has codimension $\geq 1$ in $\Sspace$. Trajectories are determined by initial conditions and dynamics. For generic $\mathcal{M}_{\mathcal{C}}$ and $\Scoord_0$, the intersection of the trajectory with $\mathcal{M}_{\mathcal{C}}$ determines a unique path. The recurrence condition $\|\gamma(T) - \Scoord_0\| < \epsilon$ selects the unique trajectory returning to initial state.
\end{proof}

\begin{corollary}[Deterministic Computation]
Poincaré computing is deterministic: identical initial conditions and constraints produce identical trajectories.
\end{corollary}

\subsection{Parallel Computation}

Multiple trajectories can evolve simultaneously in S-entropy space.

\begin{definition}[Parallel Trajectories]
A parallel computation consists of $N$ trajectories $\{\gamma_1, \ldots, \gamma_N\}$ evolving simultaneously in $\Sspace$.
\end{definition}

\begin{proposition}[Trajectory Independence]
Trajectories $\gamma_i$ and $\gamma_j$ are independent if their constraint manifolds are disjoint: $\mathcal{M}_{\mathcal{C}_i} \cap \mathcal{M}_{\mathcal{C}_j} = \emptyset$.
\end{proposition}

\begin{proof}
Disjoint constraint manifolds imply that trajectories cannot influence each other: $\gamma_i(t) \in \mathcal{M}_{\mathcal{C}_i}$ and $\gamma_j(t) \in \mathcal{M}_{\mathcal{C}_j}$ with $\mathcal{M}_{\mathcal{C}_i} \cap \mathcal{M}_{\mathcal{C}_j} = \emptyset$ ensures $\gamma_i(t) \neq \gamma_j(t)$ for all $t$.
\end{proof}

\begin{corollary}[Parallel Speedup]
$N$ independent trajectories achieve $N$-fold speedup over sequential computation.
\end{corollary}

\subsection{Cellular Implementation}

Cellular systems implement Poincaré computing through molecular dynamics.

\begin{proposition}[Molecular Trajectory]
A molecule in cellular environment traces a trajectory $\gamma_{\text{mol}}: [0,T] \to \Sspace$ determined by partition coordinates and S-entropy evolution.
\end{proposition}

\begin{proof}
Molecular state is characterized by partition coordinates $(n,\ell,m,s)$. These coordinates map to S-entropy coordinates $\Scoord = (\Sk, \St, \Se)$ through deterministic transformation. Molecular dynamics drive evolution $\Scoord(t)$, producing trajectory $\gamma_{\text{mol}}(t) = \Scoord(t)$.
\end{proof}

\begin{corollary}[Cellular Computation]
Cellular function emerges from collective molecular trajectories in S-entropy space.
\end{corollary}

The cellular state at time $t$ is the ensemble of molecular trajectories $\{\gamma_i(t)\}_{i=1}^{N}$ where $N$ is the number of molecules. Cellular computation corresponds to evolution of this ensemble toward equilibrium (recurrence) while satisfying metabolic constraints \citep{alberts2002molecular}.

\subsection{Energy Landscape}

S-entropy space admits energy landscape representation.

\begin{definition}[Energy Function]
The energy function $E: \Sspace \to \RR$ assigns energy to each S-entropy coordinate:
\begin{equation}
E(\Scoord) = \sum_{i} E_i(n_i,\ell_i,m_i,s_i)
\end{equation}
summing over all occupied partition states at $\Scoord$.
\end{definition}

\begin{proposition}[Gradient Flow]
Trajectories follow gradient descent in energy landscape:
\begin{equation}
\frac{d\Scoord}{dt} = -\nabla E(\Scoord) + \boldsymbol{\xi}(t)
\end{equation}
where $\boldsymbol{\xi}(t)$ is thermal noise.
\end{proposition}

\begin{proof}
Systems evolve toward lower energy states. The gradient $\nabla E$ points toward increasing energy. Therefore, evolution follows $-\nabla E$. Thermal fluctuations add noise $\boldsymbol{\xi}(t)$ with $\langle \boldsymbol{\xi}(t) \rangle = 0$ and $\langle \boldsymbol{\xi}(t) \boldsymbol{\xi}(t') \rangle = 2\kB T \delta(t-t')$ (fluctuation-dissipation theorem) \citep{kubo1966fluctuation}.
\end{proof}

\begin{corollary}[Equilibrium as Energy Minimum]
Equilibrium states correspond to local minima of $E(\Scoord)$ where $\nabla E = 0$.
\end{corollary}

This establishes equivalence between thermodynamic equilibrium (energy minimum) and computational equilibrium (trajectory recurrence).

