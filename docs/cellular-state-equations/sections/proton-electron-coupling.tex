\section{Proton-Electron Coupling and Membrane Scaffolding}
\label{sec:proton_electron}

\subsection{Genome Capacitor Discharge-Recharge Dynamics}

The genome-membrane circuit exhibits capacitor-like behavior with electron cascade discharge balanced by proton transport recharge.

\begin{definition}[Genome Capacitance]
\label{def:genome_capacitance}
The genome acts as a capacitor with capacitance $C_{\text{genome}} \approx 1$ pF, storing charge $Q_{\text{genome}} \approx -10^{-17}$ C.
\end{definition}

\begin{theorem}[Charge Balance Requirement]
\label{thm:charge_balance}
For sustained circuit operation, the proton transport current $I_{\text{H}^+}$ must balance the electron cascade current $I_e$:
\begin{equation}
I_{\text{H}^+} = I_e = -\frac{dQ_{\text{genome}}}{dt}
\end{equation}
\end{theorem}

\begin{proof}
The genome charge evolves as:
\begin{equation}
Q_{\text{genome}}(t) = Q_0 e^{-t/\tau_{RC}}
\end{equation}
where $\tau_{RC} = R \cdot C \approx 1~\mu$s. The electron cascade current is:
\begin{equation}
I_e = -\frac{dQ_{\text{genome}}}{dt} = \frac{Q_0}{\tau_{RC}} e^{-t/\tau_{RC}}
\end{equation}
Without proton recharge, $Q_{\text{genome}} \to 0$, collapsing the electric field. Charge balance requires $I_{\text{H}^+} = I_e$, maintaining steady-state operation.
\end{proof}

\subsection{Membrane as Electron Transport Scaffolding}

Membrane lipid composition directly controls circuit parameters through charge density modulation.

\begin{definition}[Lipid Charge Density]
\label{def:lipid_charge_density}
Lipid types exhibit characteristic surface charge densities $\sigma_{\text{lipid}}$:
\begin{align}
\sigma_{\text{PC}} &\approx -5~\text{mC/m}^2 \quad \text{(phosphatidylcholine)} \\
\sigma_{\text{PS}} &\approx -10~\text{mC/m}^2 \quad \text{(phosphatidylserine)} \\
\sigma_{\text{PE}} &\approx -7~\text{mC/m}^2 \quad \text{(phosphatidylethanolamine)} \\
\sigma_{\text{PI}} &\approx -15~\text{mC/m}^2 \quad \text{(phosphatidylinositol)} \\
\sigma_{\text{CL}} &\approx -20~\text{mC/m}^2 \quad \text{(cardiolipin)}
\end{align}
\end{definition}

\begin{theorem}[Circuit Resistance-Charge Coupling]
\label{thm:resistance_charge}
The circuit resistance $R$ varies inversely with membrane charge density:
\begin{equation}
R = \frac{k_R}{|\sigma_{\text{membrane}}|}
\end{equation}
where $k_R \approx 5 \times 10^3~\Omega \cdot \text{m}^2/\text{C}$ is an empirical constant.
\end{theorem}

\begin{proof}
Electron cascade velocity scales with charge density: $v_{\text{cascade}} \propto |\sigma_{\text{membrane}}|$. Higher velocity reduces transit time $\tau = d/v$, decreasing effective resistance $R \propto \tau$. Combining yields $R \propto 1/|\sigma_{\text{membrane}}|$.
\end{proof}

\begin{corollary}[Lipid-Dependent Conductivity]
\label{cor:lipid_conductivity}
Cardiolipin (CL) membranes exhibit $4\times$ higher conductivity than phosphatidylcholine (PC) membranes due to higher charge density.
\end{corollary}

\subsection{Lipid Physical Chemistry and Curvature}

Lipid molecular geometry determines membrane assembly and transporter insertion capability.

\begin{definition}[Packing Parameter]
\label{def:packing_parameter}
The lipid packing parameter $P$ quantifies molecular shape:
\begin{equation}
P = \frac{v}{a_0 \cdot l_c}
\end{equation}
where $v$ is tail volume, $a_0$ is head group area, and $l_c$ is tail length.
\end{definition}

\begin{theorem}[Curvature-Assembly Relationship]
\label{thm:curvature_assembly}
Lipid packing parameter determines assembly structure:
\begin{align}
P < 1 &\implies \text{Micelle (positive curvature)} \\
P = 1 &\implies \text{Bilayer (zero curvature)} \\
P > 1 &\implies \text{Inverted micelle (negative curvature)}
\end{align}
\end{theorem}

\begin{definition}[Spontaneous Curvature]
\label{def:spontaneous_curvature}
The spontaneous curvature $C_0$ characterizes preferred membrane geometry:
\begin{align}
C_0(\text{PC}) &\approx 0~\text{nm}^{-1} \quad \text{(flat bilayer)} \\
C_0(\text{PE}) &\approx -0.5~\text{nm}^{-1} \quad \text{(negative curvature)} \\
C_0(\text{PI}) &\approx +0.3~\text{nm}^{-1} \quad \text{(positive curvature)} \\
C_0(\text{CL}) &\approx -0.8~\text{nm}^{-1} \quad \text{(highly negative)}
\end{align}
\end{definition}

\begin{theorem}[Transporter Assembly Requirement]
\label{thm:transporter_assembly}
Membrane protein insertion requires negative spontaneous curvature $C_0 < 0$ with free energy:
\begin{equation}
\Delta G_{\text{insertion}} = \Delta G_{\text{hydrophobic}} + \Delta G_{\text{electrostatic}} + \kappa C_0^2
\end{equation}
where $\kappa \approx 20 k_B T$ is the bending modulus.
\end{theorem}

\begin{proof}
Protein insertion creates local membrane deformation. Negative curvature lipids (PE, CL) stabilize this deformation through curvature matching, reducing $\Delta G_{\text{insertion}}$. Optimal PE fraction of 30-40\% minimizes insertion energy while maintaining bilayer stability.
\end{proof}

\begin{corollary}[Mitochondrial CL Enrichment]
\label{cor:mitochondrial_cl}
Mitochondrial membranes are enriched in cardiolipin (CL) due to dual requirements: (1) high conductivity for electron transport, (2) negative curvature for cristae formation.
\end{corollary}

\subsection{Geometric Aperture Selection}

Proton transporters operate as geometric apertures, not Maxwell demons~\cite{flatt2023abc}.

\begin{definition}[Geometric Aperture Radius]
\label{def:aperture_radius}
Proton transporters exhibit aperture radius $r_{\text{aperture}} \approx 1.4$ \AA, permitting proton passage while excluding larger ions.
\end{definition}

\begin{theorem}[Geometric Selectivity]
\label{thm:geometric_selectivity}
Passage probability through geometric aperture scales as:
\begin{equation}
P_{\text{passage}} = \begin{cases}
\left(\frac{r_{\text{particle}}}{r_{\text{aperture}}}\right)^2 & \text{if } r_{\text{particle}} < r_{\text{aperture}} \\
0 & \text{if } r_{\text{particle}} \geq r_{\text{aperture}}
\end{cases}
\end{equation}
\end{theorem}

\begin{proof}
Geometric passage requires particle radius $r_{\text{particle}} < r_{\text{aperture}}$. The passage cross-section scales as $\pi r_{\text{particle}}^2$, while the aperture cross-section is $\pi r_{\text{aperture}}^2$. The ratio yields $P_{\text{passage}} \propto (r_{\text{particle}}/r_{\text{aperture}})^2$ for $r_{\text{particle}} < r_{\text{aperture}}$.
\end{proof}

\begin{corollary}[Proton Selectivity]
\label{cor:proton_selectivity}
For proton radius $r_{\text{H}^+} \approx 0.88$ fm and aperture radius $r_{\text{aperture}} \approx 1.4$ \AA:
\begin{align}
P_{\text{H}^+} &\approx 1 \quad \text{(essentially 100\% passage)} \\
P_{\text{Na}^+} &= 0 \quad \text{(blocked, } r_{\text{Na}^+} \approx 1.16~\text{\AA)} \\
P_{\text{K}^+} &= 0 \quad \text{(blocked, } r_{\text{K}^+} \approx 1.52~\text{\AA)} \\
P_{\text{Ca}^{2+}} &= 0 \quad \text{(blocked, } r_{\text{Ca}^{2+}} \approx 1.14~\text{\AA)}
\end{align}
\end{corollary}

\begin{remark}
Geometric aperture selection requires no information processing, resolving Maxwell's paradox. Selectivity arises purely from size exclusion, not thermodynamic information erasure~\cite{landauer1961irreversibility}.
\end{remark}

\subsection{Metabolic Cost of Lipid Synthesis}

Lipid composition optimization balances functional benefit against metabolic cost.

\begin{definition}[Lipid Synthesis Cost]
\label{def:lipid_cost}
ATP cost per lipid molecule:
\begin{align}
\text{Cost}_{\text{PC}} &\approx 4~\text{ATP} \\
\text{Cost}_{\text{PE}} &\approx 3.5~\text{ATP} \\
\text{Cost}_{\text{PS}} &\approx 4.5~\text{ATP} \\
\text{Cost}_{\text{PI}} &\approx 5~\text{ATP} \\
\text{Cost}_{\text{CL}} &\approx 8~\text{ATP}
\end{align}
\end{definition}

\begin{definition}[Functional Benefit]
\label{def:functional_benefit}
Functional benefit quantifies ability to form diverse structures:
\begin{equation}
B_{\text{functional}} = |C_0| + |P - 1|
\end{equation}
measuring deviation from flat bilayer (PC baseline).
\end{definition}

\begin{theorem}[Evolutionary Optimization]
\label{thm:evolutionary_optimization}
Evolutionary selection optimizes the cost-benefit ratio:
\begin{equation}
\eta_{\text{lipid}} = \frac{B_{\text{functional}}}{\text{Cost}_{\text{ATP}}}
\end{equation}
\end{theorem}

\begin{proof}
Cellular fitness depends on membrane functionality (enabling transport, signaling, compartmentalization) relative to metabolic investment. Lipid compositions with higher $\eta_{\text{lipid}}$ provide greater fitness advantage. Phosphatidylethanolamine (PE) exhibits optimal $\eta_{\text{PE}} \approx 0.43$ (high benefit, moderate cost), explaining its prevalence in biological membranes.
\end{proof}

\begin{corollary}[Physiological Lipid Mixture]
\label{cor:physiological_mixture}
Typical mammalian cell membrane composition (PC: 45\%, PE: 30\%, PS: 15\%, PI: 8\%, CL: 2\%) represents evolutionary optimization balancing structural stability (PC), transporter assembly (PE), signaling (PS, PI), and specialized function (CL).
\end{corollary}

\subsection{Phase Behavior and Membrane Dynamics}

Lipid phase transitions determine membrane fluidity and functional dynamics.

\begin{definition}[Order Parameter]
\label{def:order_parameter}
The membrane order parameter $S$ quantifies lipid organization:
\begin{equation}
S = \begin{cases}
1 & \text{gel phase (ordered, rigid)} \\
0 & \text{fluid phase (disordered, dynamic)} \\
0.5 & \text{phase transition}
\end{cases}
\end{equation}
\end{definition}

\begin{theorem}[Phase Transition Temperature]
\label{thm:phase_transition}
The melting temperature $T_m$ depends on lipid composition:
\begin{align}
T_m(\text{PC}) &\approx 270~\text{K} \\
T_m(\text{PE}) &\approx 310~\text{K} \\
T_m(\text{mixture}) &= \sum_i x_i T_m(i)
\end{align}
where $x_i$ is the mole fraction of lipid type $i$.
\end{theorem}

\begin{corollary}[Physiological Fluidity]
\label{cor:physiological_fluidity}
At physiological temperature $T = 310$ K, typical membranes exhibit $S \approx 0.2$-$0.3$ (fluid phase), enabling dynamic processes including volume oscillations and transporter conformational changes.
\end{corollary}

\subsection{Coupling to Circuit Dynamics}

Membrane properties directly couple to genome-membrane circuit performance.

\begin{theorem}[Cascade Velocity-Charge Coupling]
\label{thm:velocity_charge}
Electron cascade velocity scales with membrane charge density and temperature:
\begin{equation}
v_{\text{cascade}} = v_0 \left(1 + \beta |\sigma_{\text{membrane}}|\right) \sqrt{\frac{T}{T_0}}
\end{equation}
where $v_0 \approx 5 \times 10^5$ m/s, $\beta \approx 5 \times 10^4$ m$^3$/(C$\cdot$s), and $T_0 = 310$ K.
\end{theorem}

\begin{proof}
Electric field strength scales with charge density: $E \propto |\sigma_{\text{membrane}}|$. Electron velocity under electric field is $v \propto E$. Thermal velocity contribution adds $\sqrt{T}$ dependence. Combining yields the stated relationship.
\end{proof}

\begin{corollary}[Lipid-Dependent Circuit Timescale]
\label{cor:lipid_timescale}
The RC time constant depends on lipid composition through resistance:
\begin{equation}
\tau_{RC} = R \cdot C = \frac{k_R}{|\sigma_{\text{membrane}}|} \cdot C
\end{equation}
Higher charge density lipids (CL, PI) reduce $\tau_{RC}$, accelerating circuit dynamics.
\end{corollary}

\begin{theorem}[Optimal Lipid Composition for Circuit Performance]
\label{thm:optimal_composition}
The optimal lipid composition minimizes $\tau_{RC}$ while maintaining membrane stability and transporter assembly capability, achieved at approximately PC: 45\%, PE: 30\%, PS: 15\%, PI: 8\%, CL: 2\%.
\end{theorem}

\begin{proof}
This composition provides: (1) sufficient charge density ($\sigma \approx -8.5$ mC/m$^2$) for low resistance, (2) adequate negative curvature (30\% PE) for transporter assembly, (3) membrane stability (45\% PC baseline), (4) signaling capability (PS, PI), (5) $\tau_{RC} \approx 1$ $\mu$s matching biological timescales. Deviations reduce overall circuit performance.
\end{proof}
